wir freuen uns Ihnen mitteilen zu können, dass die Studierendenschaft, vertreten durch den Studierendenrat, eine Unterstützung Ihres Projektes beschlossen hat.
Es gilt der durch den StuRa beschlossene Finanzplan, die Finanzordung sowie die Förder- und der Kreditrichtlinie des Studierendenrates.

\begin{enumerate}[label=\Roman*]
\item \textbf{Projekt}\hfill [\usekomavar{projId}] \usekomavar{projName}
\item \textbf{Projektdauer} \hfill \usekomavar{projDauer} 
\item \textbf{StuRa Beschluss} \hfill \usekomavar{sturaBeschluss}
\item \textbf{Bewilligter Betrag} \hfill \usekomavar{sturaBetrag} EUR
\item \textbf{Davon auf Vorschuss} \hfill \usekomavar{sturaVorkasse} EUR
\end{enumerate}
Mit der bewilligten Förderung erklärt sich der Förderungsempfänger einverstanden, das Logo des Studierendenrates in allen werberelevanten Materialien zu veröffentlichen.

Bitte senden Sie uns alle Termine und weitere presserelevanten Informationen mindestens 14 Tage vor Projektbeginn an \href{mailto:ref-oeff@tu-ilmenau.de}{ref-oef@tu-ilmenau.de}.
Um den sachgemäßen Gebrauch der Fördermittel zu überprüfen, ist zwei
Mitgliedern des Studierendenrates kostenfrei Zutritt zu der/den Veranstaltung/en, z.B. mit Hilfe von Freikarten, zu gewähren.

Falls sechs Monate nach Zugang dieses Bescheides keine schriftliche und elektronische Abrechnung zu diesem Projekt beim Studierendenrat (Referat Finanzen) vorliegt, erlischt der Anspruch auf Förderung. Auf Antrag kann diese Frist aufgeschoben werden. 

