%%% Quelle: https://meinnoteblog.wordpress.com/2010/11/12/latex-vorlagen-fur-briefe-und-rechnung/
%---------------------------------------------------------------------------
\documentclass%%
%---------------------------------------------------------------------------
[fontsize=12pt,%%          Schriftgroesse
%---------------------------------------------------------------------------
% Satzspiegel
paper=a4,%%               Papierformat
enlargefirstpage=on,%%    Erste Seite anders
pagenumber=foot,%%   Seitenzahl oben mittig
%---------------------------------------------------------------------------
% Layout
headsepline=off,%%        Linie unter der Seitenzahl
parskip=half,%%           Abstand zwischen Absaetzen
firsthead=on,%%           display header on first page
firstfoot=off,%%           display footer on first page
%---------------------------------------------------------------------------
% Was kommt in den Briefkopf und in die Anschrift
fromalign=right,%%        Plazierung des Briefkopfs
fromphone=on,%%           Telefonnummer im Absender
fromrule=off,%%     Linie im Absender (aftername, afteraddress)
fromfax=on,%%             Faxnummer
fromemail=on,%%           Emailadresse
fromurl=on,%%             Homepage
fromlogo=on,%%            Firmenlogo
addrfield=on,%%           Adressfeld fuer Fensterkuverts
backaddress=on,%%         ...und Absender im Fenster
subject=untitled,%%  Plazierung der Betreffzeile
locfield=narrow,%%        zusaetzliches Feld fuer Absender
foldmarks=on,%%           Faltmarken setzen
numericaldate=off,%%      Datum numerisch ausgeben
refline=narrow,%%         Geschaeftszeile im Satzspiegel
firstfoot=on,%%           Footerbereich
%---------------------------------------------------------------------------
% Formatierung
draft=off%%                Entwurfsmodus
]{scrlttr2}
%---------------------------------------------------------------------------
\usepackage[english, ngerman]{babel}
%\usepackage{scrpage2}
%\pagestyle{scrheadings}
%\clearscrheadfoot 
\usepackage{ wasysym }
\usepackage{url}
\usepackage{lmodern}
\usepackage{enumitem}
\usepackage[utf8]{inputenc}
% symbols: (cell)phone, email
\RequirePackage{marvosym}
% for gray color in header
\RequirePackage{color}
\usepackage[T1]{fontenc}
\usepackage{graphicx}
\usepackage[pdftitle={\id}]{hyperref}
\usepackage{ifthen}
\usepackage{lastpage}
%---------------------------------------------------------------------------
% Schriften werden hier definiert
\renewcommand*\familydefault{\sfdefault} % Latin Modern Sans
\setkomafont{fromname}{\sffamily\color{mygray}\LARGE}
%\setkomafont{pagenumber}{\sffamily}
\setkomafont{subject}{\mdseries}
\setkomafont{backaddress}{\mdseries}
\setkomafont{fromaddress}{\small\sffamily\mdseries\color{mygray}}
%---------------------------------------------------------------------------
\begin{document}
    %---------------------------------------------------------------------------
    % Briefstil und Position des Briefkopfs
    \LoadLetterOption{DIN} %% oder: DINmtext, SN, SNleft, KOMAold.
    \makeatletter
    \@setplength{sigbeforevskip}{18mm} % Abstand der Signatur von dem closing
    \@setplength{firstheadvpos}{17mm} % Abstand des Absenderfeldes vom Top
    \@setplength{firstfootvpos}{265mm} % Abstand des Footers von oben
    \@setplength{firstheadwidth}{\paperwidth}
    \@setplength{locwidth}{70mm}   % Breite des Locationfeldes
    \@setplength{locvpos}{60mm}    % Abstand des Locationfeldes von oben
    \ifdim \useplength{toaddrhpos}>\z@
    \@addtoplength[-2]{firstheadwidth}{\useplength{toaddrhpos}}
    \else
    \@addtoplength[2]{firstheadwidth}{\useplength{toaddrhpos}}
    \fi
    \@setplength{foldmarkhpos}{6.5mm}
    \makeatother
    %---------------------------------------------------------------------------
    % Farben werden hier definiert
    % define gray for header
    \definecolor{mygray}{gray}{.55}
    % define blue for address
    \definecolor{myblue}{rgb}{0.25,0.45,0.75}

    %---------------------------------------------------------------------------
    % Absender Daten
    \setkomavar{fromlogo}{\includegraphics[height=3.5cm]{img/stura-try.pdf}}
    \setkomavar{fromname}{Studierendenrat der TU Ilmenau}
    \setkomavar{fromaddress}{Max-Planck-Ring 7\\98693 Ilmenau}
    \setkomavar{fromphone}[\phone~]{03677\,/\,69\,-\,1914}
    \setkomavar{fromfax}[\FAX~]{03677\,/\,69\,-\,1193}
    \setkomavar{fromemail}[\Letter~]{ref-finanzen@tu-ilmenau.de}
    \setkomavar{fromurl}[\Mundus~]{https://stura.tu-ilmenau.de}
    %\setkomafont{fromaddress}{\small\rmfamily\mdseries\slshape\color{myblue}}

    \setkomavar{backaddressseparator}{, }
    %\setkomavar{backaddress}{StuRa der TU Ilmenau, Felderhof 112, 40880 Ratingen} % wenn erwünscht kann hier eine andere Backaddress eingetragen werden
    \setkomavar{signature}{\usekomavar{hv} \\ Haushaltsverantwortliche/r \\des Studierendenrates der TU Ilmenau}
    % signature same indention level as rest
    \renewcommand*{\raggedsignature}{\raggedright}
    \setkomavar{location}{\raggedleft
    \vspace{1cm}
    %Rechnungsnummer XZY hier gut platziert
    }%%Zusätzliches Label Rechts von der Anschrift

    % Anlage neu definieren
    \renewcommand{\enclname}{Anlagen}
    \setkomavar{enclseparator}{: }
    %---------------------------------------------------------------------------
    % Seitenstil
    % pagenumber=footmiddle
    \pagestyle{myheadings}%% keine Header in der Kopfzeile bzw. plain
    \markboth{\usekomavar{fromname}}{\usekomavar{titel}}
    \pagenumbering{arabic}
    %---------------------------------------------------------------------------
    %---------------------------------------------------------------------------
    \setkomavar{firstfoot}{
    \centering{Seite \arabic{page} von \pageref{LastPage}}\\
    \footnotesize%
    \rule[3pt]{\textwidth}{.4pt} \\
    \begin{tabular}[t]{l@{}}%
        \usekomavar{fromname}\\
        \usekomavar{fromaddress}\\
    \end{tabular}%
    \hfill
    \begin{tabular}[t]{l@{}}%
        \usekomavar[\phone~]{fromphone}\\
        \usekomavar[\Letter~]{fromemail}\\
    \end{tabular}%
    \ifkomavarempty{frombank}{}{%
    \hfill
    \begin{tabular}[t]{l@{}}%
        \usekomavar{frombank}
    \end{tabular}%
    }%
    }%
    \setkomavar{nextfoot}{
    \parbox[t][1in][c]{\textwidth}{
    \centering{Seite \arabic{page} von \pageref{LastPage}}\\
    \rule[3pt]{\textwidth}{.4pt} \\
    \footnotesize
    \begin{tabular}[t]{l@{}}%
        \usekomavar{fromname}\\
        \usekomavar{fromaddress}\\
    \end{tabular}%
    \hfill
    \begin{tabular}[t]{l@{}}%
        \usekomavar[\phone~]{fromphone}\\
        \usekomavar[\Letter~]{fromemail}\\
    \end{tabular}%
    \ifkomavarempty{frombank}{}{%
    \hfill
    \begin{tabular}[t]{l@{}}%
        \usekomavar{frombank}
    \end{tabular}%
    }}}
    %---------------------------------------------------------------------------
    % Bankverbindung
    \setkomavar{frombank}{IBAN: DE88840510101113003290 \\BIC: HELADEF1ILK }
    %---------------------------------------------------------------------------
    \newkomavar{vereinName}
    \newkomavar{vereinPerson}
    \newkomavar{vereinAdresse}
    \newkomavar{vereinOrt}
    \newkomavar{datum}
    \newkomavar{projId}
    \newkomavar{projName}
    \newkomavar{projDauer}
    \newkomavar{projAbrechnungDatum}
    \newkomavar{sturaBeschluss}
    \newkomavar{sturaBetrag}
    \newkomavar{sturaVorkasse}
    \newkomavar{sturaTitel}
    \newkomavar{anlagen}
    \newkomavar{hv}
    \newkomavar{betreff}
    \newkomavar{sturaAbrechnung}
    \newkomavar{sturaRest}
    \newkomavar{iban}
    \newkomavar{titel}

    \input{parameter.tex}
    %only comment in for test purpose on latex editor
    %\newcommand{\filename}{bewilligung}\newcommand{\betreff}{Bewilligungsbescheid}
    %\ifthenelse{\equal{\filename}{auszahlung}}{\setkomavar{betreff}{Prüf- und Auszahlungsbescheid}}{}
    %\ifthenelse{\equal{\filename}{bewilligung}}{\setkomavar{betreff}{Bewilligungsbescheid}\setkomavar{anlagen}{Abgestimmter Finanzplan}}{}
    %---------------------------------------------------------------------------
    %\setkomavar{yourref}{Test}
    %\setkomavar{yourmail}{}
    %\setkomavar{myref}{}
    %\setkomavar{customer}{}
    %\setkomavar{invoice}{}
    %---------------------------------------------------------------------------
    % Datum und Ort werden hier eingetragen
    \setkomavar{date}{den \usekomavar{datum}}
    \setkomavar{place}{Ilmenau}
    %---------------------------------------------------------------------------
    \setkomavar{subject}{\Large \textbf{\usekomavar{titel}}}
    % Briefkörper bündig am Briefkopf ausrichten
    %\setlength{\oddsidemargin}{\useplength{toaddrhpos}}
    %\addtolength{\oddsidemargin}{-1in}
    %\setlength{\textwidth}{\useplength{firstheadwidth}}
    % Hier beginnt der Brief, mit der Anschrift des Empfängers
    \begin{letter}
    {
    \usekomavar{vereinName} \\
    \usekomavar{vereinPerson} \\
    \usekomavar{vereinAdresse}\\
    \usekomavar{vereinOrt}
    }%
        %---------------------------------------------------------------------------
        % Der Betreff des Briefes
        %---------------------------------------------------------------------------
        \opening{Sehr geehrte Damen und Herren,}
        \input{\filename}
        \closing{Mit freundlichen Grüßen}
        %---------------------------------------------------------------------------
        \vspace{-4mm}
        \ps{\footnotesize Gegen diesen Bescheid kann innerhalb von 30 Tagen nach Zugang Widerspruch beim Studierendenrat der Technischen Universität Ilmenau schriftlich oder zur Niederschrift eingelegt werden.}
        \vspace{5mm}
        \ifkomavarempty{anlagen}{}{\encl{\usekomavar{anlagen}}}

        %\cc{}

        %---------------------------------------------------------------------------
    \end{letter}
    %---------------------------------------------------------------------------
\end{document}
%---------------------------------------------------------------------------

