%%% Template originaly created by Karol Kozioł (mail@karol-koziol.net) and modified for ShareLaTeX use




\documentclass[a4paper,11pt]{article}

\usepackage[T1]{fontenc}
\usepackage[utf8]{inputenc}
\usepackage{graphicx}
\usepackage{xcolor}
\usepackage{ngerman}
\usepackage[tx]{sfmath}
\usepackage{calc}
\usepackage{lastpage}
\usepackage{ifthen}
\usepackage{multicol}
\renewcommand\familydefault{\sfdefault}
\usepackage{tgheros}
\usepackage{intcalc}

\usepackage{amsmath,amssymb,amsthm,textcomp}
\usepackage{enumerate}
\usepackage{multicol}
\usepackage{tikz}

\usepackage{geometry}
\geometry{total={210mm,297mm},
left=25mm,right=25mm,%
bindingoffset=0mm, top=20mm,bottom=20mm}


%\newcommand{\ID}{125}
%\newcommand{\IBAN}{DE08 1001 0010 0745 3161 07}
%\newcommand{\empfaenger}{Michael Braun (jr)}
%\newcommand{\hv}{Lukas Staab am 2017-10-20}
%\newcommand{\kv}{Lukas Staab am 2017-10-20}
%\newcommand{\projektname}{TestTestTest999}
%\newcommand{\recht}{Referat Ehrenamt/2017-01-16 (StuRa: 1234)}
%\newcommand{\betrag}{660}
%\newcommand{\posten}{B1,B1,B1,B2}
%\newcommand{\einnahmen}{0,0,0,0}
%\newcommand{\ausgaben}{10,500,50,100}
%\newcommand{\hhptitel}{0 1001 2,0 2115 2,0 2231 2,0 2231 2}
%\newcommand{\picpaths}{1/59e93cfbbea7c.pdf}

\linespread{1.3}

\newcommand{\linia}{\rule{\linewidth}{0.5pt}}

% custom theorems if needed
\newtheoremstyle{mytheor}
    {1ex}{1ex}{\normalfont}{0pt}{\scshape}{.}{1ex}
    {{\thmname{#1 }}{\thmnumber{#2}}{\thmnote{ (#3)}}}

\theoremstyle{mytheor}
\newtheorem{defi}{Definition}

%own storage
\newcount\tmpnum
\def\storedata#1#2{\tmpnum=0 \edef\tmp{\string#1}\storedataA#2\end}
\def\storedataA#1{\advance\tmpnum by1
   \ifx\end#1\else
      \expandafter\def\csname data:\tmp:\the\tmpnum\endcsname{#1}%
      \expandafter\storedataA\fi
}
\def\getdata[#1]#2{\csname data:\string#2:#1\endcsname}

% my own titles
\makeatletter
\renewcommand{\maketitle}{
\begin{center}
\vspace{2ex}
{\huge \textsc{\@title}}
\vspace{1ex}
\\
\linia\\
\@author \hfill \@date
\vspace{4ex}
\end{center}
}
\makeatother
%%%
\newcommand{\empfaenger}{Empfänger}
\newcommand{\IBAN}{IBAN}
\newcommand{\betrag}{drölf euro}
\newcommand{\projektname}{Projektname}
\newcommand{\ID}{1234}
\newcommand{\name}{Kulureller Anteil oder so}
\newcommand{\projname}{Kulureller Anteil oder so}
\newcommand{\IDproj}{1234}
\newcommand{\picpaths}{1/pdf-sample.pdf}
\storedata\datum{{1}{2}{3}}
\storedata\beschreibung{{SDaskdjksadjkasdj}{asdlaskdlasdj sajdhasdsan djasdlkjasdkjaskldj}{asdaskdnk sdjalksdj sakdjskaldjaj}}
\storedata\dateiname{{test.pdf}{2.pdf}{3.pdf}}
% custom footers and headers
\usepackage{fancyhdr}
\pagestyle{fancy}
\lhead{}
\chead{}
\rhead{}
\lfoot{Auslagenerstattung \textnumero{} \ID}
\cfoot{}
\rfoot{Seite \thepage{} von \pageref{LastPage}}
\renewcommand{\headrulewidth}{0pt}
\renewcommand{\footrulewidth}{0pt}
%
% all section titles centered and bolded
\usepackage{sectsty}
\allsectionsfont{\centering\bfseries\large}
%
% add section label
\renewcommand\thesection{}
%


%%%----------%%%----------%%%----------%%%----------%%%

\begin{document}

\title{Antrag auf Auslagenerstattung}

\author{StuRa der TU Ilmenau}

\date{\today}

\maketitle

\section{Allgemeine Angaben}
\vspace{0.3cm}
\begin{enumerate}[I]
\item \textbf{Auslagenerstattung}\hfill [\ID] \name
\item \textbf{Zugehöriges Projekt}\hfill [\IDproj] \projname
\item \textbf{Zahlungsempfänger}\hfill \empfaenger
\item \textbf{Beantragter Gesamtbetrag} \hfill\betrag{} EUR
\end{enumerate}

%Aufschlüsselung der Belege
\section{Angeheftete Belege}
\vspace{.2cm}
\begin{enumerate}[i]

\foreach \i in {1,...,3}{
\item \begin{multicols}{2}{\textbf{Belegdatum: }\getdata[\i]\datum \\ \textbf{\getdata[\i]\dateiname}}
\end{multicols}
\vspace{-.3cm}\textbf{Beschreibung: }\getdata[\i]\beschreibung
\vspace{0.5cm}
}
\end{enumerate}

\newpage
%Belege
\tikzset{
  font={\fontsize{29pt}{12}\selectfont}}
\foreach \nr/\picname in \picpaths
{
	%% bei mehrseitigen Belegen erst alle originale und dann alle Kopien 
	%% bei einseitigen Belegen, beides auf einer Seite
	
	\pdfximage{\picname}
    %Switch case der Anzahl Seiten des Beleges
    
                
    \begin{tikzpicture}[remember picture,overlay]
    	 %rechteck
         \draw [draw=black,line width=5mm,opacity=0.3] (16.5,-10) rectangle (-1,-1.25);
        	%belegnummer		
		\node [opacity=1,anchor=north west,xshift=0.95\textwidth, yshift=1.25cm] {B\nr};
			%text im Rechteck
		\node [opacity=0.3,anchor=north west, yshift=-2.5cm] {Hier unten abgebildeten };
		\node [opacity=0.3,anchor=north west, yshift=-3.6cm] {Beleg \nr{} antackern (Original)};
		\node [opacity=0.3,anchor=north west, yshift=-5.5cm] {Falls dieser Beleg ein A4 Beleg };
		\node [opacity=0.3,anchor=north west, yshift=-6.6cm] {ist, hefte das Original vor};
		\node [opacity=0.3,anchor=north west, yshift=-7.7cm] {dieser Seite ab};
         \node [anchor =north east, inner sep=0pt,outer sep=0pt,yshift=-0.45\paperheight] at (current page.north east) {\fbox{\includegraphics[page=1,angle=90,width=0.925\paperwidth,height=0.45\paperheight]{\picname}}};
	        \node [rotate=90, opacity=0.75,anchor=north west,xshift=-0.73\paperheight,yshift=2.92cm]{ Beleg \nr - Seite 1};
    \end{tikzpicture}
    \newpage
    \ifthenelse{\the\pdflastximagepages > 1}{
        	\foreach \index in {2,...,\the\pdflastximagepages}
				{%iteriere über die Seiten des Beleg pdfs
				\ifthenelse{\not \isodd{\index}}{
					\begin{tikzpicture}[remember picture,overlay]  
   						\node [opacity=1,anchor=north west,xshift=0.95\textwidth, yshift=1.75cm] {B\nr}; 
      					\node [anchor =north east, inner sep=0pt,outer sep=0pt,yshift=-2cm] at (current page.north east) {\fbox{\includegraphics[page=\index,angle=90,width=0.925\paperwidth,height=0.415\paperheight]{\picname}}};
		    			\node [rotate=90, opacity=0.75,anchor=north west,xshift=-0.275\paperheight,yshift=2.92cm] { Seite \index};
  					\end{tikzpicture}
				}{ %else unterer Teil des Blattes
					\begin{tikzpicture}[remember picture,overlay]  
      					\node [anchor =north east, inner sep=0pt,outer sep=0pt,yshift=-14.55cm] at (current page.north east) {\fbox{\includegraphics[page=\index,angle=90,width=0.925\paperwidth,height=0.415\paperheight]{\picname}}};
		    			\node [rotate=90, opacity=0.75,anchor=north west,xshift=-0.65\paperheight,yshift=2.92cm] { Seite \index};
  					\end{tikzpicture}
					\newpage
				} %if end
   		  		
		}{} %else end if page > 1

    \newpage
	}
}


\end{document}
